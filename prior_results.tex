\section{Results from Prior NSF Support}
%Either here or in related work: 
%PI HAS EXPERIENCE BREAKING/ANALYZING EXISTING DEFENSES (PHISHPRINT ETC.)


The PI is an investigator on the project titled: \emph{Collaborative Research: SaTC: CORE: Medium: Defending Against Social Engineering Attacks with In-Browser AI} (Award \#~2422035: \$399,979, October 1, 2021 - September 30, 2025). 
This is a project in collaboration with University of Georgia and Stony Brook University. 
\underline{\emph{Intellectual Merit:}} This project proposed to create a comprehensive framework for discovering, studying, and defending against generic web-based social engineering attacks. PI's involvement so far led to multiple publications that can help in improving our knowledge about existing social engineering attacks and how to combat them with more under active peer review~\cite{ozen2024senet}. \underline{\emph{Broader Impacts:}} As planned in the project, the PI has taught a new undergraduate/graduate course dedicated to web security that was well received by students. The PI also participated in outreach efforts to disseminate advice on social engineering attacks via talks and media interviews. The most recent effort was a town hall meeting done at a state-level in collaboration with AARP Louisiana that reached more than 200,000 people statewide. \underline{\emph{Publications:}} ~\cite{LiuPVP23,SubramaniMSVP22,ChaliseNV22,AcharyaV22,honeytweets,cframe_oakland}


\textbf{RELATION TO THIS GRANT.} While this work focused entirely on developing browser-based defense solutions, we realized that only a small part of the scam actually happens here. Considering TSS as an example, the scam involves might involve web browser or email, phone, and then remote desktop sharing. There is very little in-browser AI can help given the robust spread of channels. What is limiting to develop end-to-end defenses is the lack of data to realize this, which is what we are trying to do here. 
