\section{Thrust-1: Scam baiter studies}
\label{sec:thrust_one}


POINTS TO BE MADE (NO ORDER):

\begin{itemize}

\item First read this properly~\cite{SahinRF17} to understand guidelines for Lenny-like bots (Sec 6.5) which will help in motivating questions well.

\item HAVE A FIG SHOWING: observational studies (using think-a-loud protocol), interview studies based on remembering, passive data analysis of forum posts, technical analysis of products on vigilante markets and mapping them to state-of-the-art research. 

\item some stats about scammer info groups: how many posts per day, how many participants. just gives a gist of the amount of data we expect to gain from all this analysis. 

\item PI already has experience utilizing passive data for defensive insights~\cite{tasr} (although this was offensive attack knowledge). 

\item Weave this into text somewhere: REGARDING SE, pretty much most of the work is on phishing in web - both measurements as well as user studies, defense etc. There's a strong reason for this. THEY HAVE LOTS OF DATA WE DON'T AND THIS MAKES IT DIFFICULT TO PURSUE DEFNESIVE MEASURES. More deeply here: In web-based credential phishing attacks, there exist a steady influx of SE attacks that are blanket-sent to everyone. Thus, they end up getting populated in blocklists. Same is not the case for human-in-the-loop attacks due to the hindrances involved in being able to gather data. natural evasive factors that disable this. As a result, most research is only using artificially generated data.   Discuss psychological factors (MOVED TO THRUST-1). 

\item clarify that for the observational studies, we will be mindful of the ethical boundaries. We are going to seek a separate IRB approval for this and establish its parameters very clearly. 

\item this is like a mutually beneficial thing - for each thing we find, we will try to establish a delta with sota research and make a comparative analysis (either emiprically or theoretically). Thus, while the research community will be kept abreast of what is happening in baiting community the baiting community will know what is happening in the academic world as wel so that both are on the same page. 


\item In \cite{honeytweets}, we manually talked to scammers and found out they move from platform to platform where each part of the interaction is done in one. We found out that this has evasion benefits as no one platform has the full picture of the scam. Further, see this from CG regarding psych benefits of platform hopping (some points are likely non-sensical, but still...):
Scammers frequently direct their victims across multiple online platforms — for example, starting with a message on a social media site, then shifting to encrypted messaging apps, and ultimately leading victims to fraudulent websites. This progression is not arbitrary; it exploits well-established psychological principles. Chief among them is the principle of commitment and consistency (Cialdini, 1984): once a victim complies with an initial request, they are more likely to comply with subsequent ones to remain behaviorally consistent. This technique also mirrors the foot-in-the-door strategy, where small initial acts of compliance increase the likelihood of more significant concessions later. Moreover, this cross-platform navigation creates a sense of escalation of commitment — victims who have already invested time and attention are less likely to disengage, even as red flags emerge. Finally, platform-hopping serves a tactical function by disrupting familiar cues and bypassing trust and safety mechanisms, leaving users more vulnerable in less familiar or less protected digital environments. Understanding and modeling these dynamics is essential for designing data-driven interventions that can detect, disrupt, or preempt such adversarial behavioral sequences.

\item Note: when writing above note that Allodi's group also applied these psych principles to discuss phishing attacks ~\cite{HeijdenA19}. This probably should be weaved in somewhere.

\item So, in the end, finding out about platform hopping helps directly in offering security benefits as we figure out who all need to collaborate beyond our own planned benefit of being able to design the subsequent stages of our pipeline. 


\textbf{Text from DHS grant}

Scam baiters are modern-day vigilantes who voluntarily engage with TSS
scammers to waste their time and disrupt their operations. Each baiting call they make helps protect
potential victims and often enables them to gather incriminating evidence, which they later report to law
enforcement. Therefore, our primary motivation in this task is to conduct studies with the scam baiting
community to gain deeper insights into TSS scams and the scam baiting process. We plan on conducting
three kinds of studies with the scam baiting community which we describe below. PI Jasim who is an
expert in using HCI techniques for creating systems to address complex socio-technical challenges [20,
21, 22] will lead this task and guide the two graduate students.


First, we conduct observational studies, in which our primary objective is to examine scam baiting calls
made by participants and extract insights. We plan to recruit individuals from online scam baiting
communities [12, 23], specifically those residing in “one-party consent states” in the U.S. [24] enabling
us to legally record their interactions with tech support scammers. During each participant session, two
graduate students will document preliminary observations of the technical setups used in scam baiting.
Key questions guiding this process include: (1) What hardware is employed for scam baiting? (2) What is
the VoIP? (3) Do participants alter their voices for different calls? (4) Do they modify their voices to
appear more appealing to scammers (e.g., sounding like older adults), and if so, how? (5) How do they
configure their virtual machines (VMs) to appear legitimate to scammers? (6) Do they change phone
numbers between calls, and if so, by what method? Preliminary responses to these and other related
questions will be obtained through observation. Additionally, analyzing the recorded call transcripts will
allow us to construct a comprehensive linguistic model of the scammers’ commands and instructions.
This model will be instrumental in Task-3, where AI components will be developed to autonomously
interpret scammer utterances and execute appropriate actions on the VM.


Furthermore, insights from the observational studies will inform the development of a structured
interview protocol for one-on-one interviews with participants. These in-depth interviews will provide an
opportunity to explore scam baiting techniques and past experiences that may not be fully captured in a
single observation session. The interviews will follow a semi-structured format, allowing flexibility for
follow-up questions as new aspects of scam baiting emerge. For both observation and interview studies,
we will conduct inductive thematic analysis and continue recruiting participants until we reach saturation
in our identified themes. We anticipate interviewing approximately 30 participants for each study type.
The findings will guide the selection of components for the AI-driven scam baiter in Task-3. To ensure
the most effective design, we will collaborate with scam baiters in a co-design workshop to create a
blueprint for the AI system. Involving the scam baiters in the co-design process will enable us to leverage
their expertise and incorporate them into our proposed interventions, reducing translational effects and
mitigate information loss and interpretation errors.


\end{itemize}