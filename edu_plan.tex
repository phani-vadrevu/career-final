\section{Education plan}
\label{sec:edu_plan}


\BfPara{EDUCATION PLAN:} FIG IDEA!! HAVE A WIRE FRAME DIAGRAM TO HELP READERS VISUALIZE HOW OUR TOOL WILL LOOK LIKE. PERHAPS, MADHU CAN HELP HERE.

\begin{enumerate}

\item to help in the design stage of this, we will collaborate with a commercial anti-vishing education company that offers services to organizations
\item In this work, talking to the scammers helped them find out that they ask victims to not tell the clerks that they are here for Tech support payments as this will result in a large tax. This is an evasion tactic that only came out after talking to scammers. \cite{ayumu2024threat}   
\item Using data from Thrust-2 as GT, try to develop a scam baiter. Going into some technical details if needed. For example: (1) We can leverage the html snippets from the T2. (2) we can leverage scammer actions like mouse movements what they are clicking to simulate similar actions again on our site. (3) if they build custom sites, we can do the same. 
\item We are going to show exactly end-to-end everything that happens and make it a fun sofware. For example in one part of site there can be phone and in another there can be desktop.
\item IMPORTANT POINT: Since everything is DRIVEN by the scammer, this is a much easier problem to handle compared to Thrust-2 as we have the script with us already and we are not going to deviate from the script (like deviate from the vocab etc.) We can of course reuse the models from above (llm, action models etc.) to make this more ``interactive'' and thus engaging. 
\item One more point: 
\item We need to replicate things like like turning off remote control of the machine etc. 
\item To save time, we can make the user go through one scenario end-to-end and then let the user do another scenario from the milestone: x/N if the path only diverges from that point.
\item IMPORTANT POINT: WE PLAN TO BUILD THIS AS A FRAMEWORK SO OTHERS CAN EXTEND THIS IN THE FUTURE!! IT IS DIFFICULT TO DO EVERYTHING.
\item From ~\cite{Wash20} on phishing training approaches. This is something we can draw on for our section here. There exist 3 methods: ``general-purpose training messages that communicate best practices; fake phishing campaigns ~\cite{WashC18}; and in-the-moment warning messages~\cite{PetelkaZS19}''

\end{enumerate}





Below ideas are dumb. Consider above ones!

\begin{enumerate}
    \item ROUGH IDEA -undergrads: make them take part in a workshop/class to do a switching from scam-baiting to attack exercise? Don't know if this is a good one. But, this helps undergrads to gain knowledge about se attacks that older adults get exposed to while giving them ml/sec experience.  Cite reseach that you found that says kids are the best - but that they don't have time.
    \item release them as assignments that others can use. exposure helps people to help others out.
\end{enumerate}